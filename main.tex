\documentclass[12pt,a4paper,lmag,nt,article,french]{docDTN}

\usepackage{parskip}
\usepackage{lineno,hyperref}
\modulolinenumbers[5]
\usepackage{amsmath}
\usepackage{graphicx,caption,color,epstopdf}
%\usepackage{pst-all}
\usepackage{makeidx}
\usepackage{multirow}
\usepackage{xspace}
\usepackage{tabularx}
\usepackage{rotating}
\usepackage{float}
\usepackage{amsmath,amssymb,pifont}
\usepackage[mathscr]{eucal}
%\usepackage{natbib}
%\usepackage[english,french]{babel}
\usepackage[utf8x]{inputenc}
\usepackage{tikz}
\usepackage{glossaries}
\usepackage[off]{auto-pst-pdf}
\usepackage{lipsum}
\usepackage{pdfpages}
\usepackage{mhchem}
\usepackage{amsthm}
\usepackage{placeins}


\renewcommand\arraystretch{1.5}

\raggedbottom

\makeatletter
\def\ps@pprintTitle{
      \let\@oddhead\@empty
      \let\@evenhead\@empty
      \def\@oddfoot{\reset@font\hfil\thepage\hfil}
      \let\@evenfoot\@oddfoot
}
\makeatother

% \setlength{\parindent}{10pt}

\bibliographystyle{elsarticle-num}


%%%%%%%%%%%%%%%%%%%%%%%%%%%%%%%%%%%%%%
%% Definition du Titre et auteur(s) %%
%%%%%%%%%%%%%%%%%%%%%%%%%%%%%%%%%%%%%%
\settitre{Modèle de croûte en périphérie du bain de corium en fond de cuve dans PROCOR v2.3}
\setauteurfonction{Mathieu~\textsc{Peybernes}}{Ingénieur de recherche}{DTN/SMTA/LMAG}
\setauteurfonction{Louis~\textsc{Viot}}{Ingénieur de recherche}{DTN/SMTA/LMAG}
\setemail{mathieu.peybernes@cea.fr}
%% 
%% Facultatif
%% 
%% |  PAGE 1: PAGE DE GARDE
%%%%%%%%%%%%%%%%%%%%%%%%%%%%%%%%%%%%%%%%%%%%%%%%%
%% Definition du Numero reference du document  %%
%%%%%%%%%%%%%%%%%%%%%%%%%%%%%%%%%%%%%%%%%%%%%%%%%
\setnumero{2018-024}
%% |  PAGE 2: Fiche Documentaire 
%%%%%%%%%%%%%%%%%%%%%%%%%%%%%%%%%%%%%%%%%
%% Definition du type de diffusion     %%
%%  normale, restreinte, CEA, CD ou SD %%
%%%%%%%%%%%%%%%%%%%%%%%%%%%%%%%%%%%%%%%%%
 \setdiffusion{normale} %%%% normale | restreinte | CEA | CD | SD
%%%%%%%%%%%%%%%%%%%%%%%%%%%%%%%%%%%%%%%%%%%%%%%%%%%%%%%%%%%%%%%%%
%% Definition du type de partenaires, accord et type d'action  %%
%% pour le cadre de la page 2 (3 champs ci-dessous)            %%
%%%%%%%%%%%%%%%%%%%%%%%%%%%%%%%%%%%%%%%%%%%%%%%%%%%%%%%%%%%%%%%%%
 \setreferencesaction{EDF-AREVA}{CEA-EDF-AREVA - F18192}{}
%%%%%%%%%%%%%%%%%%%%%%%%%%%%%%%%%%%%%%%%%%%%%%%%%%%%%%%%%%%%%%%%%
%% Definition des references internes CEA                      %%
%% pour le cadre de la page 2 (4 champs ci-dessous)            %%
%%%%%%%%%%%%%%%%%%%%%%%%%%%%%%%%%%%%%%%%%%%%%%%%%%%%%%%%%%%%%%%%%
 \setreferencesinterne{DISN}{GEN 2 \& 3}{CORIU}{A-CORIU-03-03-11}                 
%%%%%%%%%%%%%%%%%%%%%%%%%%%%%%%%%%%%%%%%%%%%%%%%%%%%%%%%%%%%
%% Definition d'un jalon (oui ou non) et de son intitule  %%
%% pour le cadre de la page 2 (2 champs ci-dessous)       %%
%%%%%%%%%%%%%%%%%%%%%%%%%%%%%%%%%%%%%%%%%%%%%%%%%%%%%%%%%%%%
 %\setjalon{XXX}{XX}       
%%
%% Indices 
%%  
%%%%%%%%%%%%%%%%%%%%%%%%%%%%%%%%%%%%%%%%%%%%%%%%
%% Definition de la date du document indice A %%
%% pour le cadre de la page 2                 %%
%%%%%%%%%%%%%%%%%%%%%%%%%%%%%%%%%%%%%%%%%%%%%%%%
\setindiceAdate{xx/12/2018}       
%%%%%%%%%%%%%%%%%%%%%%%%%%%%%%%%%%%%%%%%%%%%%%%%%%%%%%%%%%%%%%%%
%% Definition de la date du document et nature de l'evolution %%
%% de l'indice B pour le cadre de la page 2                   %%
%%%%%%%%%%%%%%%%%%%%%%%%%%%%%%%%%%%%%%%%%%%%%%%%%%%%%%%%%%%%%%%%
%\setindiceB{15/06/2010}{Mise en conformité ....}
%%
%%%%%%%%%%%%%%%%%%%%%%%%%%%%%%%%%%%%%%%%%%%%%%%%%%%%%%%%%%%%%%%%
%% Definition des Signatures pour le cadre de la page 2       %%
%%  Redacteur(s) : automatique a partir des auteurs           %%
%%  Verificateur                                              %%
%%  Autre Visa (controle QSE)                                 %% 
%%  Approbateur                                               %%
%%  Emetteur                                                  %%
%%                 Deux champs : nom et fonction              %%
%%%%%%%%%%%%%%%%%%%%%%%%%%%%%%%%%%%%%%%%%%%%%%%%%%%%%%%%%%%%%%%%
%% Signatures  
%%
\setverificateur{Romain {\sc Le TEllier}}{Ingénieur de recherche}
\setverificateur{Sylvie {\sc Dubois}}{Chef de projet}
\setapprobateur{Laurent {\sc Saas}}{Chef de Laboratoire}
\setemetteur{Dominique {\sc Pêcheur}}{Chef de Service}            
%%
%% |  PAGE 3: RESUME        
%%%%%%%%%%%%%%%%%%%%%%%%%%%%%%%%%%%%%%%%%%%%%
%% Definition des mots clefs de la page 3  %%
%%%%%%%%%%%%%%%%%%%%%%%%%%%%%%%%%%%%%%%%%%%%%
\setmotscles{corium en cuve, stratification, instabilités de Rayleigh-Taylor, champ de phase, CALPHAD}
%%%%%%%%%%%%%%%%%%%%%%%%%%%%%%%%%%%%%%%%%%%%%
%% Definition du resume de la page 3       %%
%%%%%%%%%%%%%%%%%%%%%%%%%%%%%%%%%%%%%%%%%%%%%
\setresume{
Ce document reprend le premier livrable de la prestation menée par le laboratoire PMC de l'Ecole Polytechnique pour le laboratoire DTN/SMTA/LMAG du CEA Cadarache intitulée ``Développement d'un modèle à interface diffuse de diffusion multicomposant pour la simulation de la stratification d'un bain de corium oxyde-métal en cuve d'un réacteur nucléaire à eau légère''. Cette note est préalable au travail de développement en tant que tel qui sera mené, en grande partie, par un stagiaire post-doctoral au laboratoire PMC. Sont présentés ici les équations des modèles \textit{a priori} mis en \oe uvre, les questions scientifiques associées et le plan de travail relatif à ce développement. 
}
%%%%%%%%%%%%%%%%%%%%%%%%%%%%%%%%%%%%%%%%%%%%%
%% Definition du resume DO de la page 3    %%
%%%%%%%%%%%%%%%%%%%%%%%%%%%%%%%%%%%%%%%%%%%%%

%% |  PAGE 4: DIFFUSION INTERNE CEA
%%%%%%%%%%%%%%%%%%%%%%%%%%%%%%%%%%%%%%%%%%%%%%%%%%%%
%% Definition de la diffusion interne/externe CEA %%
%% (par exemple sous le forme d'un tableau)       %%
%%                                                %%
%% La base documentaire SIBIL est automatiquement %%
%% ajoutée                                        %%
%% De meme pour la GED DER 03 04.16.01.03 si DO   %%
%%%%%%%%%%%%%%%%%%%%%%%%%%%%%%%%%%%%%%%%%%%%%%%%%%%%
\setdiffusioninterne{
     \begin{tabular}[!H]{ll}
    \underline{Document complet}:\\
    CEA/DEN/CAD/DTN/SMTA/LMAG & Tous \\
    %CEA/DEN/CAD/DTN/SMTA/LEAG & J.~DELACROIX, A.~PIVANO, P.~PILUSO \\
    %CEA/DEN/DANS/DPC/SCCME/LM2T & C.~GUENEAU, A.~QUAINI  \\
    CEA/DEN/CAD/DTN/SMTA & D.~PECHEUR, C~.VALOT \\
    CEA/DTN & S.~DUBOIS \\
    CEA/DISN & P.~DUMAZ
\\
%     CEA/DEN/CAD/DPIE/SA2P & J. BROCHARD
% \\  
    EDF/7N & G.~FLEURY, B.~TOURNIAIRE pour distribution EDF/7N \\
    EDF Lab & K.~ATKHEN, A.~LE BELGUET pour distribution EDF Lab \\
    Framatome & C.~CARDON, M.~LECOMTE pour distribution Framatome 
\\
    \\
    \underline{Copie}:\\
    CEA/DTN/CAD/DTN/DIR & C.~DELLIS, C.~TRUFFIER \\
    CEA/DTN/CAD/DTN/STCP & O.~GASTALDI \\                                                   
    CEA/DEN/CAD/DTN/SMTA/LEAG & C.~SUTEAU \\                                                 
    CEA/DEN/CAD/DTN/SMTA/LMCT & J. P.~FERAUD \\                                                       
    CEA/DEN/CAD/DTN/SMTA/LMTE & F.~JOURDAIN \\                                                  
    CEA/DEN/CAD/DTN/SMTA/LMN & F.~JALLU \\ 
    CEA/DEN/DANS/DPC/SCCME/LM2T & S.~GOSSE \\
    EDF & E.~SAUVAGE
\\
    Framatome & A.~CAILLAUX
\\
\end{tabular}
}
%% 
%% YES/NO
%% 
%%%%%%%%%%%%%%%%%%%%%%%%%%%%%%%%%%%%%%%%%%%%%%%%%
%% Definition d'un document                    %%
%% sans liste des figures (\nolistoffigures)   %%
%% sans liste des tables (\nolistoftables)     %%
%% sans timbre de diffusion (\notimbre)        %%
%%%%%%%%%%%%%%%%%%%%%%%%%%%%%%%%%%%%%%%%%%%%%%%%%
%\nolistoffigures
\nolistoftables
%\notimbre
%%%%%

  \begin{document}

\section{Introduction}
\section{Modélisation}
\subsection{modèle de croûte}
Le domaine 2D occupé par la croûte est noté $\Omega_C$. Les propriétés physiques associées à cette croûte sont : $T_C$ la température, $\rho_C$ la densité, $\lambda_C$ la conductivité thermique, $C_{p,C}$ la capacité thermique, $\Delta {\cal H}^{fus}$ l'enthalpie de fusion et $T_{fus}$ (resp. $T_{sol}$) la température de fusion de la croûte (resp. la température de solidification du corium liquide).


On note $\gamma_{in}$ la partie de la frontière interne du domaine $\Omega_C$ (i.e. côté bain de corium liquide) et $\gamma_{ext}$ la partie de la frontière externe du domaine $\Omega_C$ (i.e. côté cuve), voir figure \ref{fig:crust_figure}. Sur la frontière $\gamma_{in}$, une partie de la croûte est en contact avec le corium liquide, elle est alors soumise à un flux $\phi_{bain}$ provenant du bain de corium liquide. Sur la frontière $\gamma_{ext}$, une condition de température imposée ($T_C=T^{out}$) ou de flux imposé ($\phi=\phi_{out}$) peut être définie.

\begin{figure}[H]
\centering
\includegraphics[width=0.7\textwidth]{Figures/crust_figure.png}
\caption{Schéma de la croûte couplée à au bain de corium liquide et de la cuve} \label{fig:crust_figure}
\end{figure}

Le domaine $\Omega_C$ est discrétisé en 1D selon la direction verticale (selon l'axe noté $z$) de la croûte. Le maillage est décrit par $N_{noeuds}$ noeuds. Les $N_{mailles}=N_{noeuds}-1$ mailles rectangulaires correspondantes sont notées $C_j$ pour $1 \leq j\leq N_{mailles}$. Pour chaque maille, la direction horizontale est donnée par l'axe $x$. On note [$x_j^{min}, x_j^{max}$] le domaine occupé par la maille $j$ suivant l'axe $x$ et $S_j$ la surface de la face verticale (i.e. suivant l'axe $z$). A chaque pas de temps, le bain de corium peut changer de configuration (e.g. stratification, baisse ou élévation du niveau du bain). Ainsi, à chaque pas de temps le domaine occupé par la croûte est remaillé à partir d'un pas d'espace de référence en respectant les contraintes suivantes :
\begin{itemize}
    \item une maille ne peut pas être en face de deux couches du bain, ainsi celle-ci ne recevra un flux thermique provenant uniquement d'une seul couche
    \item quelque soit l'épaisseur d'une couche du bain de corium, au moins une maille sera definie en face de cette couche
    \item  Il peut apparaître éventuellement au cours d'un calcul une autre partie de $\gamma_{in}$ située plus haut que le niveau supérieure du bain de corium liquide (e.g. lorsque le niveau du bain baisse, une partie de la croûte ne voit plus le corium liquide). Dans ce cas on considérera une condition adiabatique.
\end{itemize}
A chaque remaillage du domaine occupé par la croûte, une projection des masses et des températures est effectuée sur le nouveau maillage en assurant une conservation en masse et en énergie.

Les quantités moyennes suivantes sont introduites pour $1 \leq j\leq N_{mailles}$ :
\begin{eqnarray}
m_{j}(t) &=& \rho_C  V_j(t) \\
\overline{T}_{j}(t) &=& \frac{1}{V_j(t)} \int_{C_j(t)} T_{j}(x,z,t)\,\mathrm{d}x\, \mathrm{d}z\\
\overline{\phi}_{j,in}(t) &=& -\frac{1}{S_j}\int_{\gamma_{in}\cap S_j}\lambda_C \frac{\partial T_{j}}{\partial x}(x=x_j^{min},z,t)\, \mathrm{d}z  \label{eq:phi_j_in}\\
\overline{\phi}_{j,ext}(t) &=& -\frac{1}{S_j}\int_{\gamma_{in}\cap S_j}\lambda_C \frac{\partial T_{j}}{\partial x} (x=x_j^{max},z,t)\, \mathrm{d}z  \label{eq:phi_j_ext}
\end{eqnarray}
avec $T_{j}$ la température de la maille $j$ et $V_j$ le volume de la maille $j$.\\

L'évolution des masses et températures moyennes pour chaque maille $j$, $\left(m_{j}, \overline{T}_{j}\right)_{j\in[1, N_{mailles}]}$ est calculée sur un pas de temps entre $t$ et $t+\Delta t$ en résolvant le système :

\begin{eqnarray*}
\frac{dm_{j}}{dt} &=& - \frac{dm_{j}^\text{fus/sol}}{dt} \\ 
m_{j} C_{p,C} \frac{d \overline{T}_{j}}{dt} &+& \frac{dm_{j}^\text{fus/sol}}{dt}C_{p,C} \left(T^{fus/sol} - \overline{T}_{j}\right) \\ &=& S_i\left(\overline{\phi}_{j,in} - \overline{\phi}_{j,ext}\right) + \dot{q}^m m_{j}
\end{eqnarray*}

où :

\begin{itemize}
 \item $\frac{dm_j^\text{fus/sol}}{dt}$ est le taux de masse ablatée dans le cas d'une fusion de la croûte ou le taux de masse solidifiée dans le cas d'une solidification du corium liquide. Ce taux de masse est positif dans le cas d'une fusion;
 \item $\dot{q}^m m$ représente la contribution de la puissance résiduelle (W) associée à la puissance résiduelle massique $\dot{q}^m$(W/kg);
\end{itemize}
Dans ce système, la conduction axiale (i.e. suivant l'axe $z$) a été négligée. Il est à noter qu'une option est disponible dans le code pour prendre en compte une approximation numérique de flux thermique axial (\cite{NENE2018}). 

Une maille $C_j$ peut être dans trois types "d'état" : soit elle est en fusion, soit en solidification, soit le volume de la maille $C_j$ reste constant. Dans ce dernier cas, on dira par la suite que la maille $C_j$ est en "pure conduction". Par la suite, en considérant que $T_{sol}\leq T_{j,in}\leq T_{fus}$, on supposera qu'une maille $C_j$ doit passer par un état de "pure conduction" pour passer d'un état de fusion à un état de solidification ou inversement.

En fonction de la température de la frontière interne de la croûte $\gamma_{in}$, que l'on note $T_{j,in}$, les fermetures associées au système d'équations précédent sont différentes. Ci-dessous sont listés les différents cas de figure :\\

{\it cas d'une pure conduction dans la croûte}\\
Tant que $T_{j,in}<T_{fus}$, la fusion de la maille de croûte $C_j$ n'a pas commencé :
\begin{eqnarray*}
\frac{dm_j^\text{fus/sol}}{dt} &=& 0.0 \\
\overline{\phi}_{j,in} &=& \phi_{j,bain}
\end{eqnarray*}

{\it passage en fusion de la croûte}\\
Si $T_{j,in}$ atteint la température $T_{fus}$ ($T_{j,in}\ge T_{fus}$), la fusion de la maille $C_j$ commence suivant l'équation de front de fusion suivante :
\begin{eqnarray*}
\Delta \mathcal{H}^{fus} \frac{dm_j^\text{fus/sol}}{dt} &=& S_j\left(\phi_{j,bain} - \overline{\phi}_{j,in}\right) \\
T_{j,in} &=& T^{fus}
\end{eqnarray*}

{\it passage en solidification du corium}\\
Si $T_{j,in}$ atteint la température $T^{sol}$ ($T_{j,in}\le T^sol$), la solidification du corium commence suivant l'équation du front de solidification suivante :
\begin{eqnarray*}
\Delta \mathcal{H}^{fus} \frac{dm_j^\text{fus/sol}}{dt} &=& S_j\left(\phi_{j,bain} - \overline{\phi}_{j,in}\right) \\
T_j^{in} &=& T^{sol}
\end{eqnarray*}

{\it passage en conduction pure après un arrêt d'une fusion/solidification}\\
Lorsque une maille $C_j$ est en fusion ou solidification, si $\frac{dm_i^\text{fus/sol}}{dt}$ s'annule, le front de fusion/solidification s'arrête. Et ainsi la maille $C_j$ devient en "conduction pure" :
\begin{eqnarray*}
\frac{dm_j^\text{fus/sol}}{dt} &=& 0.0 \\
\phi_j^{in} &=& \phi_{j,bain}
\end{eqnarray*}


En pratique, selon l'état de la maille $C_j$, il est nécessaire de compléter également l'équation d'énergie par des lois de de fermeture pour les quantités $\overline{\phi}_{j,in}$ and $\overline{\phi}_{j,ext}$. Pour ce faire, un modèle 0D basé sur l'hypothèse d'un profil quadratique suivant l'axe $x$ de la température permet de déduire ces $\overline{\phi}_{j,in}$ and $\overline{\phi}_{j,ext}$ \cite{LeTellier2016}.

\subsection{Couplage avec le bain de corium}
\section{Résultats numériques}
\section{conclusion}
\end{document}
