Dans le cadre de l'amélioration de la modélisation du comportement du corium en fond de cuve, ce rapport propose un modèle permettant la prise en compte explicite d'une croûte située à l'interface d'un bain de corium avec la cuve. Pour ce faire, un maillage 1D de la croûte est introduit pour la résolution des équations d'évolution en masse et énergie. Le déplacement du front de fusion/solidification est traité par un modèle de Stefan. En fonction de la température à l'interface entre le bain et la croûte, l'algorithme développé permet à chaque maille de croûte d'être soit en fusion, soit en solidification, soit dans un état dans lequel seule la conduction est considérée lorsque la température d'interface se situe entre une température de solidification et de fusion. La température de fusion est donnée par la température de liquidus associée à la composition de la croûte, tandis que la température de solidification est donnée par la température de liquidus associée à la composition de la couche du bain qui est en contact. 

En effet, avec ce modèle, on cherche à s'affranchir des hypothèses de macro-ségrégation du modèle initial de PROCOR qui sont problématiques lors de l'inversion de stratification (formation d'une couche métallique légère au-dessus du bain oxyde) puisqu'elles conduisent à considérer que la couche métallique légère se retrouve toujours en face d'une croûte réfractaire.

En fonction du changement de configuration du bain (stratification, hauteur), un remaillage est réalisé afin de permettre un couplage pertinent entre la croûte et le bain assurant un transfert thermique rigoureux entre les mailles de croûte et les couches du bain. Les masses et les énergies associées au nouveau maillage sont assurées par une projection conservative de ces quantités. Le couplage en temps est assuré par la plateforme PROCOR à l'aide d'une représentation par graphe des modèles et communications de données (flux thermiques, températures, débits de masse).\\

Une application numérique, de l'ordre du test de vérification du modèle, est présentée dans ce document pour illustrer le couplage entre le modèle de croûte et celui du bain de corium. Les résultats obtenus permettent de vérifier le bon comportement du modèle dans le cas de la solidification à l'interface d'un bain homogène oxyde. Lors de l'apparition  subséquente d'une phase métallique en contact avec la croûte oxyde (associée au transitoire de stratification), de par le choix d'une modélisation distinguant des températures de solidification et de fusion différentes à l'interface, le cas test présenté a mis en lumière des oscillations non-physiques associées au schéma de couplage explicite et, eventuellement, aux fermetures du bilan thermique de la croûte (approximation des flux conductifs). Cette pathologie requiert une analyse plus détaillée afin de pouvoir fournir une première version satisfaisante du modèle couplé avec un bain stratifié. Elle sera réalisée par le biais d'un cas test proposé dans cette note. \\

La suite directe du travail rapporté ici est le test de ce modèle sur des applications réacteurs et la comparaison avec les résultats obtenus avec le modèle de bain initial de PROCOR avec une croûte ``fictive''. Ce travail sera réalisé en 2019 dans le cadre du projet IVMR avec une reprise des benchmarks étudiés dans le WP2.2 et une nouvelle série de calculs réacteurs dans le WP2.5.

Une fois cette étape réalisée, la modélisation de la dissolution de la croûte par l'acier liquide pourra alors être abordée sur la base des résultats du travail post-doctoral de A. Pivano réalisé dans le cadre du WP3.2 du projet IVMR (essais VITI-CORMET).

En perspectives du travail de modélisation, une évolution du modèle proposé dans ce rapport est le traitement de la conduction dans la croûte en 2D (et non plus seulement radiale). Ce développement se justifie lorsque l'épaisseur de croûte devient suffisamment importante pour que la conduction dans la direction axiale puisse avoir un effet. Ce cas de figure peut arriver aux temps longs, lorsque le bain de corium se refroidit significativement et que la masse de solide devient importante, mais également aux temps courts où la croûte est plus épaisse dans la partie inférieure du fond de cuve. Le travail d'analyse des expériences de thermohydraulique des bains (e.g. LIVE) qui sera initié en 2019 dans le cadre de la fiche ``Modélisation des accidents graves'' du projet CORIU permettra de fournir des éléments de validation de ce modèle de croûte et, le cas échéant, de préciser les conditions dans lesquelles la conduction 2D devient importanet à prendre en compte.

Par ailleurs, une étude complémentaire sera menée afin d'évaluer l'approximation dans les codes 0D consistant à prendre une température moyenne à l'interface de la couche mince métallique située en surface (température moyenne typiquement prise à la température de fusion de la cuve). En effet, lorsque la configuration du bain évolue, une portion de croûte issue de la solidification de la couche oxyde peut se retrouver sur une partie de l'interface de la couche métallique. Le bain de corium peut ainsi être à la fois en contact avec la croûte sur une partie de l'interface et avec la cuve sur une autre portion. Dans le cadre du projet IVMR, des études avec un code de CFD seront réalisées pour estimer l'effet de cette variation de température à l'interface.
