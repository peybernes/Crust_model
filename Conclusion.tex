Dans ce rapport, une modélisation d'une croûte située à l'interface d'un bain de corium avec la cuve est proposée. Pour ce faire, un maillage 1D de la croûte est introduit pour la résolution des équations d'évolution en masse et énergie. Le déplacement du front de fusion/solidification est traité par un modèle de Stefan. En fonction de la température à l'interface entre le bain et la croûte, l'algorithme développé permet à chaque maille de croûte d'être soit en fusion, soit en solidification, soit dans un état où seule la conduction est considérée lorsque la température d'interface se situe entre une température de solidification et de fusion. La température de fusion est donnée par la température de liquidus associée à la composition de la croûte, tandis que la température de solidification est donnée par la température de liquidus associée à la composition de la couche du bain qui est en contact. 

En fonction du changement de configuration du bain (stratification, hauteur), un remaillage est réalisé afin de permettre un couplage pertinent entre la croûte et le bain afin d'assurer un transfert thermique rigoureux entre les mailles de croûte et les couches du bain. Les masses et les énergies associées au nouveau maillage sont assurées par une projection conservative de ces quantités.

Les résutlats numériques obtenus ...\\

En perspectives, une évolution du modèle proposé dans ce rapport est le traitement de la conduction dans la croûte en 2D (et non plus seulement radiale). Ce développement se justifie lorsque l'épaisseur de croûte devient suffisamment épaisse pour que la conduction dans la direction axiale puisse avoir un effet. Ce cas de figure peut arriver aux temps longs, lorsque le bain de corium se refroidit et se solidifie au contact de la cuve. Mais également aux temps courts où la présence d'une croûte plus épaisse dans la partie inférieure du fond de cuve peut être envisagée. L'interprétation prévue en 2019 des essais LIVE, qui mettent en évidence la présence d'une croûte qui s'épaissit sur la surface inférieure de la cuve, permettra de mettre en évidence ce type d'études.   