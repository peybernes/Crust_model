La présente note répond à un jalon COB. Dans le cadre de l’amélioration de la modélisation du comportement du corium en fond de cuve, un modèle permettant la prise en compte explicite d’une croûte située à l’interface d’un bain de
corium avec la cuve est proposé. Pour ce faire, un maillage 1D dynamique (avec des remaillages cohérents vis-à-vis des possibles changements de stratification du bain) de la croûte est introduit pour la résolution des équations d’évolution en masse et énergie. Le déplacement du front de fusion/solidification est traité par un modèle de Stefan. Le couplage en temps est assuré par la plateforme PROCOR. Des applications numériques sont proposées afin de vérifier le couplage entre le modèle de croûte et celui du bain de corium. Les résultats obtenus permettent de vérifier le bon comportement du modèle dans le cas de la solidification à l'interface d'un bain homogène oxyde. Lors de l'apparition  subséquente d'une phase métallique en contact avec la croûte oxyde (associée au transitoire de stratification), de par le choix d'une modélisation distinguant des températures de solidification et de fusion différentes à l'interface, le cas test présenté a mis en lumière des oscillations non-physiques associées au schéma de couplage explicite et, eventuellement, aux fermetures du bilan thermique de la croûte (approximation des flux conductifs). Cette pathologie requiert une analyse plus détaillée afin de pouvoir fournir une première version satisfaisante du modèle couplé avec un bain stratifié. Cette analyse sera réalisée par le biais d'un cas test proposé dans cette note afin d'étudier plus particulièrement le schéma en temps et l'approximation du modèle de conduction.

Ce modèle, dont le développement a été réalisé dans le cadre de la fiche ``plateforme PROCOR'' du projet CORIU pour les années 2017 et 2018, sera intégré à la version v2.3 de la plateforme PROCOR. Sous réserve de la correction du problème d'oscillations, il pourra alors être utilisé pour des applications réacteurs.
