Dans le cadre de l’amélioration de la modélisation du comportement du corium en fond de cuve, cette note propose un modèle permettant la prise en compte explicite d’une croûte située à l’interface d’un bain de
corium avec la cuve. Pour ce faire, un maillage 1D dynamique (avec des remaillages cohérents vis à vis des possibles changements de stratification du bain) de la croûte est introduit pour la résolution des équations d’évolution en masse et énergie. Le déplacement du front de fusion/solidification est traité par un modèle de Stefan. Le couplage en temps est assuré par la plateforme PROCOR. Des applications numériques sont proposées afin de vérifier le couplage entre le modèle de croûte et celui du bain de corium. Les résultats obtenus permettent de vérifier le comportement en transitoire et en stationnaire de la croûte et de s’assurer de la conservation en masse et en énergie. Enfin, un cas test est détaillé pour permettre de travailler sur les artefacts numériques et problèmes de modélisation identifiés lors des applications numériques.

Le développement de ce modèle, qui s’inscrit dans le cadre de la fiche ``plateforme PROCOR'' du projet CORIU pour les années 2017 et 2018, sera intégré à la version v2.3 de la plateforme PROCOR. 
